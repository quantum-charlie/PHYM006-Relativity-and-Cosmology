%
% Report Template Version 1.0
% University of Exeter
% Department of Physics and Astronomy
%
%
% Comments start with % (percent) character and last till the end of the line.
%
% Compile this latex file using pdflatex command, rather than the latex 
% command, to produce pdf output directly  
%
% LaTeX2e document starts with \documentclass[options]{<class-name>}
%
% <class-name> can be one of the standard LaTeX document classes: 
% article, report or book, or some other specialised class.
%
% options: a4paper - paper size; onecolumn or two column format; 12pt font size
%
% always use a minimum of a 12pt font throughout for readability 
%
% uncomment/comment the appropriate documentclass of 2 options below
%
%\documentclass[a4paper,onecolumn,12pt]{article}
\documentclass[a4paper,twocolumn,11pt]{article}
%
% Preamble of LaTeX document is everything before \begin{document}.
%
% Preamble is used to load extension packages and to set up global 
% parameters and configuration for the entire document.
%
% Extension packages providing additional functionality:
%

\usepackage{amsmath}      % additional math environments
\usepackage{graphicx}     % graphics import from external files 
\usepackage{caption}      % customisation of captions
\usepackage{booktabs}     % table typesetting additions
\usepackage{url}          % format url addresses
\usepackage{abstract}	  % allows formatting of abstract
\usepackage{wasysym}      % provides astronomical symbols
\usepackage{txfonts}	     % nicer roman font than the default
\usepackage{comment}	     % allows one to use \begin{comment} and \end{comment} to comment out section

\usepackage{braket}       % added 28/01/20
\usepackage{dsfont}       % added 18/12/20
\usepackage{rotating}     % added 22/12/20
\usepackage{supertabular} % added 22/12/20
\usepackage{array}        % added 22/12/20
\newcolumntype{P}[1]{>{\centering\arraybackslash}p{#1}}
\makeatletter
\renewcommand{\maketag@@@}[1]{\hbox{\m@th\normalsize\normalfont#1}}
\makeatother
\DeclareMathOperator{\sech}{sech}
\DeclareMathOperator{\csch}{csch}
\DeclareMathOperator{\arcsec}{arcsec}
\DeclareMathOperator{\arccot}{arcCot}
\DeclareMathOperator{\arccsc}{arcCsc}
\DeclareMathOperator{\arccosh}{arcCosh}
\DeclareMathOperator{\arcsinh}{arcsinh}
\DeclareMathOperator{\arctanh}{arctanh}
\DeclareMathOperator{\arcsech}{arcsech}
\DeclareMathOperator{\arccsch}{arcCsch}
\DeclareMathOperator{\arccoth}{arcCoth} 
%
% OPTIONS FOR REFERENCES
%
% If you reference by numbers (e.g. [1]), you can enter references manually, or create a BibTeX file that 
% contains all your references.  This is particularly useful if you know you will be using references again 
% and again, or if you want to keep a library of references all in one place.
%
% This line defines a logical variable `usebibtex' that specifies that you do (true) or do not want (false) to
% use BibTeX.  The name of the BibTeX file is set near the end of the document using the 
% \bibliography{bibliography} command.
%
\newif\ifUseBibTeX
%
% These two lines define the logical variable to be either true or false: comment/uncomment the one you want:
%
\UseBibTeXtrue
%\UseBibTeXfalse
%
%
% Many physics articles use numbers for references (e.g. [1])
% Most astronomy and astrophysics articles refer to other articles by Author Name and Year. (e.g. Jones 2010)
% Choose which to use below:
%
% This line defines a logical variable `RefByNum'
\newif\ifRefByNum
%
% These two lines define the logical variable to be either true or false: comment/uncomment the one you want:
%
\RefByNumtrue
%\RefByNumfalse
%
% The following lines, down to \fi (which is the end of the if section), are required depending on which type of
% referencing you choose
%
\ifRefByNum
%
% For standard reference by number, use:
%
\usepackage{cite}         % improved handling of numeric citations
\bibliographystyle{ieeetr}   % for this style the numbers are assigned in the order they are referenced in the text
%
%
\else
%
% For reference by author and year, use:  
%
\UseBibTeXfalse
\usepackage{aas_macros}
\usepackage[round]{natbib}
\setcitestyle{aysep={}}
\bibliographystyle{mn2e}
%
%      You also need to have the following files in the directory:
%         aas_macros.sty    - this contains abbreviations of many journal names
%         mn2e.bst              - this allows you to use a bibtex file to contain the full reference information
%
%      You will need to create a ".bib" file containing bibtex records for each reference.
%	For example, if you use the NASA ADS system to find papers, it can give you the bibtex entry to copy 
%	and paste into your ".bib" file.  See, for example, http://adsabs.harvard.edu/abs/2018MNRAS.475.5618B
%
%	Which gives a bibtex entry like:
%	@ARTICLE{2018MNRAS.475.5618B,
%	   author = {{Bate}, M.~R.},
%	    title = "{On the diversity and statistical properties of protostellar discs}",
%	  journal = {\mnras},
%	archivePrefix = "arXiv",
%	   eprint = {1801.07721},
%	 primaryClass = "astro-ph.SR",
%	 keywords = {accretion, accretion discs, hydrodynamics, radiative transfer, methods: numerical, protoplanetary discs, stars: formation},
%	     year = 2018,
%	    month = apr,
%	   volume = 475,
%	    pages = {5618-5658},
%	      doi = {10.1093/mnras/sty169},
%	   adsurl = {http://adsabs.harvard.edu/abs/2018MNRAS.475.5618B},
%	  adsnote = {Provided by the SAO/NASA Astrophysics Data System}
%	}
%
%. At the end of your LaTeX document, you then reference the bibliography using:
% \bibliography{bibtexfilename}
%     where your bib file has, for example, the name: bibtexfilename.bib
%
\fi
%
% END OF DEFINITIONS FOR REFERENCES
%
% PAGE FORMATTING
%
% To set margins etc.:
%
\textheight 24.0cm        % sets the length of the text on each page
\textwidth 16.0cm         % sets the width of the text on each page
\topmargin -1.25cm        % sets the top margin: - higher, + lower 
\oddsidemargin 0.5cm      % makes the left margin: + wider, - narrower
%
% \renewcommand{\abstractname}{}    % removes abstract title
%
% sets abstract margins, but no real need to do this:
% \setlength{\absleftindent}{30mm}
% \setlength{\absrightindent}{30mm}
%
% some handy commands for referencing;
% the optional argument overrides the default label, e.g.
% \figref[FIG.~]{fig:label}
\newcommand{\figref}[2][\figurename~]{#1\ref{#2}}
\newcommand{\tabref}[2][\tablename~]{#1\ref{#2}}
\newcommand{\secref}[2][Section~]{#1\ref{#2}}
%
%
% BEGINNING OF THE ACTUAL DOCUMENT
%
% The document opens with \begin{document} and closes with \end{document}
%
\begin{document}
%
% The aim of the formatting should be to optimise readability, hence reports
% should be single column and double line spaced, with an appropriate number of 
% words per line - between 8 and 12. 
% edit title and author as needed:
%
\title{\textbf{\Huge PHYM006 - Relativity and Cosmology}}           % fill in the title here
\author{\LARGE Key Equations and Concepts}         % fill in your name here
%
\date{\Large \textit{Academic Year 20/21}}  % resets date of the report from today's date
%
%\twocolumn[	          % makes title and abstract appear over entire page 
                          % width - possibly required if two-column option 
                          % chosen in documentclass - otherwise comment out
%
\maketitle                % formats the title
%
\section{Introduction and Special Relativity (Part 1)}

\subsection{Lecture 1}

\subsubsection{Gravitational Mass}
\begin{equation}
    \mathbf{F}_\text{grav.} = \frac{G\mu_1\mu_2}{r^2}
\end{equation}

\subsubsection{Intertial Mass}
\begin{equation}
    \mathbf{F}_\text{inert.} = ma
\end{equation}

\subsection{Lecture 2}

\subsubsection{Galilean Transforms}
For a frame $S'$ moving with speed $V$ along x-axis:
\begin{equation}
    x' = x - Vt
\end{equation}
\begin{equation}
    y' = y
\end{equation}
\begin{equation}
    z' = z
\end{equation}
\begin{equation}
    v_x' = v_x - V
\end{equation}

\subsubsection{Postulates of Special Relativity}
\textit{From lecture slides:}
\begin{enumerate}
    \item{``the laws of physics can be written in all inertial frames}
    \item{the speed of light in a vacuum is constant in all inertial frames''}
\end{enumerate}

\subsubsection{Lorentz Factor}
\begin{equation}
    \gamma(V) = \frac{1}{\sqrt{1 - \frac{V^2}{c^2}}}
\end{equation}
\subsubsection{Lorentz Transforms (Coordinates)}
For a frame $S'$ moving with speed $V$ along x-axis: \newline

\noindent
\textbf{Moving Frame}
\begin{equation}
    t' = \gamma(V)\bigg(t - \frac{Vx}{c^2}\bigg)
\end{equation}
\begin{equation}
    \Delta t' = \gamma(V)\bigg(\Delta t - \frac{V\Delta x}{c^2}\bigg)
\end{equation}
\begin{equation}
    x' = \gamma(V)\big(x - Vt\big)
\end{equation}
\begin{equation}
    \Delta x' = \gamma(V)\big(\Delta x - V\Delta t\big)
\end{equation}
\begin{equation}
    y' = y
\end{equation}
\begin{equation}
    \Delta y' = \Delta y
\end{equation}
\begin{equation}
    z' = z
\end{equation}
\begin{equation}
    \Delta z' = \Delta z
\end{equation}

\noindent
\textbf{Stationary Frame}
\begin{equation}
    t = \gamma(V)\bigg(t' + \frac{Vx'}{c^2}\bigg)
\end{equation}
\begin{equation}
    \Delta t = \gamma(V)\bigg(\Delta t' + \frac{V\Delta x'}{c^2}\bigg)
\end{equation}
\begin{equation}
    x = \gamma(V)\big(x' + Vt'\big)
\end{equation}
\begin{equation}
    \Delta x = \gamma(V)\big(\Delta x' + V\Delta t'\big)
\end{equation}
\begin{equation}
    y = y'
\end{equation}
\begin{equation}
    \Delta y = \Delta y'
\end{equation}
\begin{equation}
    z = z'
\end{equation}
\begin{equation}
    \Delta z = \Delta z'
\end{equation}

\subsubsection{Lorentz Transforms (Velocities)}
For a frame $S'$ moving with speed $V$ along x-axis: \newline

\noindent
\textbf{Moving Frame}
\begin{equation}
    v_x' = \frac{v_x - V}{1 - v_x\frac{V}{c^2}}
\end{equation}
\begin{equation}
    v_y' = \frac{v_y}{\gamma(V)\bigg(1 - v_x\frac{V}{c^2}\bigg)}
\end{equation}
\begin{equation}
    v_z' = \frac{v_z}{\gamma(V)\bigg(1 - v_x\frac{V}{c^2}\bigg)}
\end{equation}

\noindent
\textbf{Stationary Frame}
\begin{equation}
    v_x = \frac{v_x' + V}{1 + v_x\frac{V}{c^2}}
\end{equation}
\begin{equation}
    v_y = \frac{v_y'}{\gamma(V)\bigg(1 + v_x'\frac{V}{c^2}\bigg)}
\end{equation}
\begin{equation}
    v_z = \frac{v_z'}{\gamma(V)\bigg(1 + v_x'\frac{V}{c^2}\bigg)}
\end{equation}

\subsubsection{Time Dilation}
\begin{equation}
    \Delta t' = \gamma(V)\Delta t
\end{equation}

\subsubsection{Length Contraction}
\begin{equation}
    L' = \frac{L}{\gamma(V)}
\end{equation}

\subsubsection{The Doppler Effect}
\begin{equation}
    f_\text{detector} = f_\text{lamp}\sqrt{\frac{c - V}{c + V}}
\end{equation}

\section{Special Relativity (Part 2) and Founding Principles of GR}

\subsection{Lecture 3}

\subsubsection{Space-Time Separation}
\begin{equation}
    (\Delta s)^2 = (c\Delta t)^2 - (\Delta x)^2 - (\Delta y)^2 - (\Delta z)^2
\end{equation}
\begin{itemize}
    \item{$(\Delta s)^2 >0$: time-like separated - ``the events are causally related and there will be a frame in which the two events happen at the same place but at different times.''}
    \item{$(\Delta s)^2 =0$: light-like separated - ``the events are causally related and all observers agree that they could be linked by a light signal.''}
    \item{$(\Delta s)^2 <0$: space-like separated - ``the events are not causally related and there will be a frame in which the two events happen at the same time but at different places.''}
\end{itemize}
\begin{equation}
    (\Delta s)^2 = \sum_{\mu=0,3;\nu=0,3}\eta_{\mu\nu}\Delta X^\mu \Delta X^\nu,
\end{equation}
where $\eta_{\mu\nu}$ is the Minkowski metric.

\subsubsection{Minkowski Metric}
\begin{equation}
    \eta_{\mu\nu} = \begin{pmatrix}
    1&0&0&0\\
    0&-1&0&0\\
    0&0&-1&0\\
    0&0&0&-1
    \end{pmatrix}
\end{equation}

\subsubsection{Proper-Time Separation}
\begin{equation}
    (\Delta\tau) = \frac{(\Delta s)^2}{c^2}
\end{equation}

\subsubsection{Relativistic Momentum}
\begin{equation}
    \mathbf{p} = \gamma(V)m\mathbf{v}
\end{equation}

\subsubsection{Total Relativistic Energy}
\begin{equation}
    E = E_k + E_0 = \gamma(V)mc^2
\end{equation}

\subsubsection{Relativistic Kinetic Energy}
\begin{equation}
    E_k = (\gamma(V)-1)mc^2
\end{equation}

\subsubsection{Rest Mass Energy}
\begin{equation}
    E_0 = mc^2
\end{equation}

\subsubsection{Energy-Momentum Relation}
\begin{equation}
    E^2 = p^2c^2 + m^2c^4
\end{equation}
\subsubsection{Four-Position}
\begin{equation}
    [x^\mu] = (ct, x, y, z) = (ct, \mathbf{r})
\end{equation}

\subsubsection{Four-Velocity}
\begin{equation}
\begin{split}
    [U^\mu] &= \bigg(\frac{dx^\mu}{d\tau}\bigg)\\
    &= \bigg(c\frac{dt}{d\tau}, \frac{dx}{d\tau}, \frac{dy}{d\tau}, \frac{dz}{d\tau}\bigg)\\
    &= (\gamma(V)c, \gamma(V)\mathbf{v})
\end{split}
\end{equation}
The four-velocity is an invariant quantity: $\sum_{\mu,\nu=0}^3 \eta_{\mu\nu}U^\mu U^\nu = c^2$.

\subsubsection{Four-Acceleration}
\begin{equation}
    A^\mu = \frac{dU^\mu}{d\tau} + \Gamma^\mu_{\alpha\beta}U^\alpha U^\beta
\end{equation}

\subsubsection{Four-Momentum}
\begin{equation}
    [P^\mu] = \bigg(\frac{E}{c}, \gamma(V)m\mathbf{v}\bigg) = \bigg(\frac{E}{c}, \mathbf{p}\bigg)
\end{equation}

\subsubsection{Four-Force}
\begin{equation}
    [F^\mu] = \bigg(\frac{dP^\mu}{d\tau}\bigg) = \bigg(\frac{1}{c}\frac{dE}{d\tau},\frac{d\mathbf{p}}{d\tau}\bigg)
\end{equation}

\subsubsection{Four-Current}
\begin{equation}
    [J^\mu] = (c\rho, J_x, J_y, J_z)
\end{equation}

\subsubsection{Continuity Equation}
\begin{equation}
    \sum_{\mu=0}^3 \frac{\partial J^\mu}{\partial x^\mu} = 0
\end{equation}

\subsection{Lecture 4}

\subsubsection{Principle of Equivalence}

\textit{From lecture slides:} \newline

\noindent
\textbf{Weak}\newline

\noindent
`` Within a sufficiently localised region of spacetime adjacent to a concentration of mass, the motion of bodies subject to gravitational effects alone cannot be distinguished from the motion of bodies within a region of appropriate uniform acceleration. '' \newline

\noindent
\textbf{Strong} \newline

\noindent
`` Within a sufficiently localised region of spacetime adjacent to a concentration of mass, the physical behaviour of bodies cannot be distinguished by any experiment from the physical behaviour of bodies within a region of appropriate uniform acceleration. ''

\subsubsection{Principle of Covariance}
``All frames (including non-inertial ones) are equivalent, so physical laws should retain their form under coordinate transformations i.e. physical laws should be constructed using tensors (constructed in tensorial form).''

\subsubsection{Principle of Consistency/Correspondence}
``The new theory should account for the successful predictions of the old theory. In the case of GR then the full theory should reduce to the Newtonian result in the limit of weak fields.''

\section{Tensors}

\subsection{Lecture 5}
\subsubsection{Contravariant Components}
Vary in the opposite way to basis vectors e.g. position, velocity, and acceleration.
\begin{equation}
    dy^n = \frac{\partial Y^n}{\partial X^m}dx^m
\end{equation}

\subsubsection{Covariant Components}
Vary in the same way to basis vectors e.g. gradients of scalars.
\begin{equation}
    W_n^{(Y)} = \frac{\partial X^m}{\partial Y^n}W_n^{(X)}
\end{equation}

\subsection{Lecture 6}
\subsubsection{Metric Tensor}
\begin{equation}
    (ds)^2 = g_{mn}dX^mdX^n
\end{equation}
\textbf{Properties}
\begin{itemize}
    \item{$g_{rs}^{(Y)} = g_{mn}^{(X)}\frac{\partial X^m}{\partial Y^r}\frac{\partial X^n}{\partial Y^s}$}
    \item{$g_{ij} = g_{ji}$}
    \item{$g_{ij}b^j = b_i$}
    \item{$g^{ij}b_j = b^i$}
    \item{$g^{ij}g_{ij} = \delta_k^i$}
\end{itemize}

\subsection{Lectures 7 and 8}

\subsubsection{Tensor Scaling}
\begin{equation}
    U_\alpha^\mu = ST_\alpha^\mu
\end{equation}

\subsubsection{Tensor Addition}
\begin{equation}
    S_\alpha^\mu + T_\alpha^\mu = U_\alpha^\mu
\end{equation}

\subsubsection{Tensor Multiplication}
\begin{equation}
    A_{\alpha\beta}^\mu = X^\mu Y_\alpha Z_\beta
\end{equation}

\subsubsection{Tensor Contraction}
\begin{equation}
    B_\gamma = A_{\sigma\gamma}^\sigma
\end{equation}

\subsubsection{Trace}
\begin{equation}
    R = R_i^i
\end{equation}

\subsubsection{Reciprocal Basis Vectors}
\begin{equation}
    \mathbf{a} = a^i\mathbf{e}_i = a_i\mathbf{e}^i
\end{equation}
\textbf{Contravariant Components}
\begin{equation}
    \mathbf{a}\cdot\mathbf{e}_i = a^j\mathbf{e}_j\cdot\mathbf{e}_i = a^j\delta_{ij} = a^i
\end{equation}
\textbf{Covariant Components}
\begin{equation}
    \mathbf{a}\cdot\mathbf{e}^i = a_j\mathbf{e}^j\cdot\mathbf{e}^i = a_j\delta^{ij} = a_i
\end{equation}

\subsection{Lecture 9}

\subsubsection{Christoffel Symbols}
\begin{equation}
    \Gamma_{ij}^m = \Gamma_{ji}^m = \frac{1}{2}g^{mk}\bigg[\frac{\partial g_{ki}}{\partial X^j} + \frac{\partial g_{jk}}{\partial X^i} - \frac{\partial g_{ij}}{\partial X^k}\bigg]
\end{equation}

\subsubsection{Covariant Differentiation}
\begin{equation}
    \nabla_\beta V^\alpha = \frac{\partial V^\alpha}{\partial X^\beta} + \Gamma_{\lambda\beta}^\alpha V^\lambda
\end{equation}
\begin{equation}
    \nabla_\beta V_\alpha = \frac{\partial V_\alpha}{\partial X^\beta} - \Gamma_{\alpha\beta}^\lambda V_\lambda
\end{equation}

\subsubsection{Geodesic Equation}
\begin{equation}
    \frac{d^2Y^n}{ds^2} + \Gamma_{mr}^n\frac{dY^m}{ds}\frac{dY^n}{ds} = 0,
\end{equation}
where $s$ is an affine parameter.

\section{Curvature and Einstein's Field Equation}

\subsection{Lecture 10}

\subsubsection{Riemann Tensor}
\begin{equation}
    R_{ijk}^l = \frac{\partial\Gamma_{ik}^l}{\partial X^j} - \frac{\partial\Gamma_{ij}^l}{\partial X^k} + \Gamma_{ik}^m\Gamma_{mj}^l - \Gamma_{ij}^m\Gamma_{mk}^l
\end{equation}

\subsubsection{Ricci Tensor}
\begin{equation}
    R_{\alpha\beta} = R_{\alpha\beta\gamma}^\gamma
\end{equation}

\subsubsection{Ricci Scalar}
\begin{equation}
    R_\alpha^\alpha = g^{\alpha\beta}R_{\alpha\beta}
\end{equation}

\subsection{Lectures 11 and 12}

\subsubsection{Einstein Tensor}
\begin{equation}
    G_{\mu\nu} = R_{\mu\nu} - \frac{1}{2}g_{\mu\nu}R
\end{equation}

\subsubsection{Einstein's Field Equations}
\begin{equation}
    R_{\mu\nu} - \frac{1}{2}g_{\mu\nu}R = -kT_{\mu\nu}
\end{equation}

\subsection{Lecture 13}

\subsubsection{Einstein's Field Equation Constant}
\begin{equation}
    k = \frac{8\pi G}{c^4}
\end{equation}

\section{Schwarzschild Metric}

\subsection{Lecture 14}

\subsubsection{Schwarzschild Metric Assumptions}
\textit{From lecture slides:}
\begin{enumerate}
    \item{``Solution is asymptotically flat''}
    \item{``Metric coefficients do not depend on time i.e. the metric is stationary''}
    \item{``Metric is static and not rotating i.e. no cross terms''}
\end{enumerate}

\subsubsection{Schwarzschild Metric}
\begin{equation}
\begin{split}
    (ds)^2 &= \bigg(1-\frac{2GM}{c^2 r}\bigg)(cdt)^2 - \frac{1}{1-\frac{2GM}{c^2 r}}(dr)^2\\
    &- r^2(d\theta)^2 - r^2\sin^2\theta(d\phi)^2
\end{split}
\end{equation}
\begin{equation}
    g_{\mu\nu} = \begin{pmatrix}
    1 - \frac{2GM}{c^2r}&0&0&0\\
    0&-\frac{1}{1 - \frac{2GM}{c^2 r}}&0&0\\
    0&0&-r^2&0\\
    0&0&0&-r^2\sin^2\theta
    \end{pmatrix}
\end{equation}

\subsubsection{Schwarzschild Radius}
\begin{equation}
    R_s = \frac{2GM}{c^2}
\end{equation}

\subsection{Lecture 15}

\subsubsection{Gravitational Time Dilation}
\begin{equation}
    d\tau_\infty = \frac{d\tau_\text{em}}{\sqrt{1 - \frac{2GM}{c^2r_\text{em}}}}
\end{equation}

\subsubsection{Gravitational Redshift}
\begin{equation}
    f_\infty = f_\text{em}\sqrt{1 - \frac{2GM}{c^2r_\text{em}}}
\end{equation}

\subsubsection{Proper Distance}
\begin{equation}
    d\sigma = \frac{dr}{\sqrt{1 - \frac{2GM}{c^2 r}}},
\end{equation}
showing that coordinates in GR have no metrical significance.

\section{Geodesics and Black Holes}

\subsection{Lecture 16}

\subsubsection{Geodesic in the Schwarzschild Metric}

\footnotesize
\begin{equation}
    \bigg(\frac{dr}{d\tau}\bigg)^2 + \frac{J^2}{m^2r^2}\bigg(1 - \frac{2GM}{c^2 r}\bigg) - \frac{2GM}{r} = c^2\bigg[\bigg(\frac{E}{Mc^2}\bigg)^2 - 1\bigg]
\end{equation}
\normalsize
For a radial path, we set $J = 0$.

\subsection{Lecture 17}

\subsubsection{Orbital Shape Equation}
\begin{equation}
    \frac{d^2U}{d\phi^2} + U = \frac{GMm^2}{J^2} + \frac{3GMU^2}{c^2},
\end{equation}
where $U = 1/r$.

\subsubsection{Relativistic Potential}
\begin{equation}
    v_\text{eff, rel} = \frac{J^2}{2m^2r^2}\bigg(1 - \frac{2GM}{c^2 r}\bigg) - \frac{GM}{r}
\end{equation}

\subsubsection{Newtonian Potential}
\begin{equation}
    v_\text{eff, new} = \frac{J^2}{2m^2r^2} - \frac{GM}{r}
\end{equation}

\subsubsection{Radius of the Last Stable Orbit}
\begin{equation}
    r = \frac{6GM}{c^2} = 3R_s
\end{equation}

\subsection{Lecture 18}

\subsubsection{Time to Reach Singularity}
\begin{equation}
    \tau_\text{sing.} = \frac{\pi r_0^{3/2}}{2cR_s^{1/2}}
\end{equation}

\subsubsection{Time to Reach Event Horizon}
\begin{equation}
    \tau_\text{horiz.} = \frac{r_0^{3/2}}{cR_s^{1/2}}\bigg[\frac{\pi}{2} - \frac{2}{3}\bigg(\frac{R_s}{r_0}\bigg)^{3/2}\bigg]
\end{equation}

\subsubsection{Time Difference for Photon Signal}
\begin{equation}
    t_2 - t_1 = \frac{r_2 - r_1}{c} + \frac{R_s}{c}\ln\bigg|\frac{r_2 - R_s}{r_1 - R_s}\bigg|,
\end{equation}
where $t_1$ and $r_1$ correspond to the emitter, and $t_2$ and $r_2$ correspond to the observer.

\section{Kerr Metric for Spinning Black Holes}

The second vacuum solution, which is valid for a vanishing stress-energy tensor. Applicable to rotating black holes.

\subsection{Lecture 19}

\begin{equation}
\begin{split}
    (ds)^2 &= \bigg(1- \frac{R_sr}{\rho^2}\bigg)(cdt)^2 - \frac{2R_sra\sin^2\theta}{\rho}dtd\phi\\
    & -\rho^2\frac{(dr)^2}{\Delta} - \rho^2(d\theta)^2\\
    &- \bigg[(r^2 + a^2)\sin^2\theta + \frac{R_sra^2\sin^4\theta}{\rho^2}\bigg](d\phi)^2
\end{split}
\end{equation}
This is written in Boyer-Lindquist coordinates, where:
\begin{itemize}
    \item{$x = \sqrt{r^2 + a^2}\sin\theta\cos\phi$}
    \item{$y = \sqrt{r^2 + a^2}\sin\theta\sin\phi$}
    \item{$z = r\cos\theta$}
    \item{$R_s = 2GM/c^2$}
    \item{$a = J/Mc$}
    \item{$\Delta = r^2 - R_sr + a^2$}
    \item{$\rho^2 = r^2 + a^2\cos^2\theta$}
\end{itemize}

\subsubsection{Physical Singularity}
\begin{equation}
    \rho = 0
\end{equation}

\subsubsection{Inner and Outer Event Horizon Singularities}

\begin{equation}
    \Delta = 0,
\end{equation}
which results in:
\begin{equation}
    r = \frac{R_s}{2} \pm \sqrt{\frac{R_s^2}{4} - a^2}.
\end{equation}

\subsubsection{Extremal Black Hole}
Maximum possible $J$ for a given $M$:
\begin{equation}
    a = \frac{R_s}{2},
\end{equation}
i.e. $r_+ = r_-$.

\subsubsection{Metric of the 2-Surface}
\begin{equation}
    g = \begin{pmatrix}
    \rho_+^2 & 0\\
    0 & \big(\frac{R_sr_+}{\rho_+}\big)^2
    \end{pmatrix}
\end{equation}

\subsubsection{Equator-Pole Circumference Ratio}
\begin{equation}
    \frac{L}{L'} \approx 1.65,
\end{equation}
where $L$ is the equator circumference, and $L'$ is the pole circumference.

\subsubsection{Frame Dragging}
From a zero angular momentum observer (ZAMO):
\begin{equation}
    \frac{d\phi}{dt} = \frac{g_{\phi t}}{g_{tt}}.
\end{equation}
So, spacetime is probably skewed in the rotational direction.

\subsubsection{Ergosphere}
\textit{From lecture slides}: ``a surface within which no time-like four velocities are possible for a stationary observer.''
\begin{equation}
    r_e = \frac{R_s}{2} \pm \sqrt{\frac{R_s}{2} - a^2\cos^2\theta}
\end{equation}

\section{Gravitational Waves}

\subsection{Lecture 20}

\subsubsection{Weak-Field Metric}
\begin{equation}
    g_{\mu\nu} = \eta_{\mu\nu} + h_{\mu\nu}
\end{equation}

\subsubsection{Linearised Vacuum Equation}
\begin{equation}
    \Box h_{\mu\nu} - \partial_\mu v_\nu - \partial_\nu v_\mu = 0,
\end{equation}
where $\Box$ is the d'Alembertian operator.

\subsubsection{Gravitational Wave Wavevector}
\begin{equation}
    k^\mu = \bigg(\frac{\omega}{c}, k\bigg)
\end{equation}

\subsubsection{Polarisation Tensor}
\begin{equation}
    h(t,z) = \begin{pmatrix}
    0&0&0&0\\
    0&f_+(ct-z)&f_X(ct-z)&0\\
    0&f_X(ct-z)&f_+(ct-z)&0\\
    0&0&0&0
    \end{pmatrix}
\end{equation}

\subsection{Lecture 21}

\subsubsection{Strain from Gravitational Waves}
\begin{equation}
    L(t) = \bigg(1 + \frac{1}{2}h_11(t,0)\bigg)L_0
\end{equation}

\subsection{Lecture 22}

\subsubsection{Gravitational Wave Luminosity}
\begin{equation}
    L = \frac{128G^{7/3}}{5c^2}4^{2/3}M^{4/3}\mu^2\bigg(\frac{\pi}{P}\bigg)^{10/3
    }
\end{equation}

\section{Cosmology and the Robertson-Walker Metric}

\subsection{Lecture 23}

\subsubsection{Cosmological Principle}
\textit{From Lambourne}: ``at any given time, and on a sufficiently large scale, the Universe is homogeneous and isotropic.''

\subsection{Lecture 24}

\subsubsection{The Robertson-Walker Metric}

\small
\begin{equation}
    (ds)^2 = (cdt)^2 - R^2(t)\bigg[\frac{(dr)^2}{1 - kr^2} + r^2((d\theta)^2 + \sin^2\theta(d\phi)^2)\bigg],
\end{equation}
\normalsize
where:
\begin{itemize}
    \item{$k$ is the curvature parameter $\{-1,0,1\}$}
    \item{R(t) is the cosmic scale factor}
    \item{t is the cosmic time}
    \item{$\{r, \theta, \phi\}$ are the comoving coordinates}
\end{itemize}

\subsubsection{Hubble's Law}
\begin{equation}
    v_p = \frac{1}{R}\frac{dR}{dt}d_p,
\end{equation}
where $v_p$ is the proper radial velocity, and $d_p$ is the proper distance.

\subsubsection{Hubble's Parameter}
\begin{equation}
    H(t) = \frac{1}{R}\frac{dR}{dt}
\end{equation}

\subsubsection{Density Scaling with $R(t)$}
\begin{equation}
    \rho_m(t) = \rho_{m,0}\bigg[\frac{R_0}{R(t)}\bigg]^3
\end{equation}
\begin{equation}
    \rho_r(t) = \rho_{m,0}\bigg[\frac{R_0}{R(t)}\bigg]^4
\end{equation}
\begin{equation}
    \rho_\Lambda(t) = \rho_{\Lambda,0}
\end{equation}

\subsubsection{Cosmic Density}
\begin{equation}
    \rho(t) = \rho_{m,0}\bigg[\frac{R_0}{R(t)}\bigg]^3 + \rho_{r,0}\bigg[\frac{R_0}{R(t)}\bigg]^ + \rho_\Lambda
\end{equation}

\subsubsection{Cosmic Pressure}
\begin{equation}
    p(t) = \frac{\rho_{r,0}c^2}{3}\bigg[\frac{R_0}{R(t)}\bigg]^4 - \rho_\Lambda c^2
\end{equation}

\section{The Cosmological Constant and the Friedmann Equations}

\subsection{Lecture 25}

\subsubsection{Field Equation with the Cosmological Constant}
\begin{equation}
    R_{\mu\nu} - \frac{1}{2}Rg_{\mu\nu} + \Lambda g_{\mu\nu} = -kT_{\mu\nu}
\end{equation}

\subsubsection{Friedmann Equations}
\begin{equation}
    \frac{1}{R}\ddot{R} = -\frac{4\pi G}{3}\bigg(\rho + \frac{3p}{c^2}\bigg)
\end{equation}
\begin{equation}
    \bigg(\frac{\dot{R}}{R}\bigg)^2 = \frac{8\pi G\rho}{3} - \frac{kc^2}{R^2}
\end{equation}

\subsubsection{Fluid Equation}
\begin{equation}
    \frac{d\rho}{dt} + \bigg(\rho + \frac{p}{c^2}\bigg)\frac{3}{R}\frac{dR}{dt} = 0
\end{equation}

\subsubsection{de Sitter Model}
$k=0$; $\rho_{m,0}=\rho_{r,0}=0$
\begin{equation}
    R(t) = R_0\exp\bigg[\sqrt{\frac{8\pi G\rho_\Lambda}{3}}(t-t_0)\bigg]
\end{equation}
\begin{equation}
    H(t) = \sqrt{\frac{8\pi G\rho_\Lambda}{3}}
\end{equation}

\subsubsection{Flat, Pure Radiation Model}
$k=0$; $\rho_{m,0}=\rho_\Lambda=0$
\begin{equation}
    R(t) = R_0\sqrt{2H_0t}
\end{equation}
\begin{equation}
    H_0 = \sqrt{\frac{8\pi G\rho_{r,0}}{3}}
\end{equation}
\begin{equation}
    H(t) = \sqrt{\frac{8\pi G\rho_{r,0}}{3}}
\end{equation}

\subsubsection{Einstein-de Sitter Model}
$k=0$; $\rho_{r,0}=\rho_\Lambda=0$
\begin{equation}
    R(t) = R_0\bigg(\frac{3}{2}H_0t\bigg)^{2/3}
\end{equation}
\begin{equation}
    H_0 = \sqrt{\frac{8\pi G\rho_{m,0}}{3}}
\end{equation}
\begin{equation}
    H(t) = \sqrt{\frac{8\pi G\rho_{m,0}}{3}}
\end{equation}

\subsubsection{Critical Density}
\textit{From lecture slides}: ``in the absence of dark energy, this is the density required to halt expansion after infinite cosmic time.''
\begin{equation}
    \rho_c(t) = \frac{3H^2}{8\pi G}
\end{equation}

\subsubsection{Density Parameters}
\begin{equation}
    \Omega_m(t) = \frac{\rho_m(t)}{\rho_c(t)}
\end{equation}
\begin{equation}
    \Omega_r(t) = \frac{\rho_r(t)}{\rho_c(t)}
\end{equation}
\begin{equation}
    \Omega_\Lambda(t) = \frac{\rho_\Lambda}{\rho_c(t)}
\end{equation}

\subsubsection{FRW Models}

\textbf{Open Universe}
\begin{equation}
    \Omega_m + \Omega_r + \Omega_\Lambda < 1 \implies k<0
\end{equation}
\noindent
\textbf{Flat Universe}
\begin{equation}
    \Omega_m + \Omega_r + \Omega_\Lambda = 1 \implies k=0
\end{equation}
\textbf{Closed Universe}
\begin{equation}
    \Omega_m + \Omega_r + \Omega_\Lambda > 1 \implies k>0
\end{equation}

\subsubsection{Redshift and Scale Factor}
\begin{equation}
    1 + z = \frac{R(t_\text{ob})}{R(t_\text{ob})}
\end{equation}

\section{Inflation}

\subsection{Lecture 26}

\subsubsection{Planck Length}
\begin{equation}
    \ell_p = \sqrt{\frac{\hbar G}{c^3}}
\end{equation}

\subsubsection{Planck Time}
\begin{equation}
    t_p = \sqrt{\frac{\hbar G}{c^5}}
\end{equation}

\subsubsection{Horizon Problem}

\textit{From lecture slides}: ``CMB is very isotropic which implies that the universe must have been in thermal equilibrium, but opposite regions of the CMBR sphere cannot have been connected by a light ray and so cannot have come into thermal equilibrium.''

\subsubsection{Flatness Problem}

\textit{From lecture slides}: ``$|\Omega_m + \Omega_r + \Omega_\Lambda - 1|$ is an increasing function of time. As such, a flat geometry is unstable and the universe must have been very flat at early times.''

\subsubsection{Inflation}

\textit{From lecture slides}: ``Rapid, exponential growth of the cosmic scale factor in the universe.''

\subsubsection{Universe Temperature}
\begin{equation}
    T = T_0\bigg(\frac{R_0}{R}\bigg)
\end{equation}

\clearpage
\appendix

\section{Maths Appendix}
\subsection{Matrices}

\noindent
\textbf{Transpose}
\begin{equation}
    A_{ij}^T = A_{ji}
\end{equation}

\noindent
\textbf{Trace}
\begin{equation}
    tr(\mathbf{A}) = \sum{A_{ii}}
\end{equation}

\noindent
\textbf{Adjoint Matrix} \newline

\noindent
A matrix of minors, $\alpha_{ij}$, with signs attached: \newline
\begin{equation}
    \text{adj}(A) = 
    \begin{pmatrix}
    + & - & + \\
    - & + & - \\
    + & - & + \\
    \end{pmatrix}
\end{equation}

\noindent
\textbf{Systems of Linear Equations} \newline

\noindent
Can be represented as:
\begin{equation}
    \textbf{AX} = \textbf{B},
\end{equation}

\noindent
and can be solved by two methods: matrix inversion and Cramer's rule. \newline

\noindent
\textbf{Matrix Inversion Method}
\begin{equation}
    \mathbf{A}^{-1} = \frac{\text{adj}(A)}{|A|}
\end{equation}

\noindent
\textbf{Cramer's Rule}
\begin{equation}
    x_i = \frac{|\mathbf{C}(i)|}{|\mathbf{A}|}
\end{equation}

\noindent
\textbf{Matrix Eigenvalue Equation}
\begin{equation}
    \mathbf{A}\mathbf{r} = \lambda \mathbf{r}
\end{equation}
\noindent
Eigenvalues can be found by taking the determinant:
\begin{equation}
    |\mathbf{A} - \lambda \mathbf{I}| = 0.
\end{equation}

\subsection{Trigonometry}

\noindent
\textbf{Double-Angle Formulae}
\begin{equation}
    sin(\theta\pm\phi)=\sin\theta\cos\phi \pm \sin\phi\cos\theta
\end{equation}
\begin{equation}
    \sin(2\theta)=2\sin\theta\cos\theta
\end{equation}
\begin{equation}
    \cos(\theta\pm\phi)=\cos\theta\cos\phi \mp \sin\theta\sin\phi
\end{equation}
\begin{equation}
    \cos(2\theta)=\cos^2\theta-\sin^2\theta
\end{equation}
\begin{equation}
    \tan(\theta\pm\phi)=\frac{\tan\theta \pm \tan\phi}{1\mp\tan\theta\tan\phi}
\end{equation}
\begin{equation}
    \tan 2\theta=\frac{2\tan\theta}{1-\tan^2\theta}
\end{equation}

\noindent
\textbf{Hyperbolic Identities}
\begin{equation}
    \sinh x =\frac{e^x-e^{-x}}{2}=i\sin x
\end{equation}
\begin{equation}
    \cosh x =\frac{e^x+e^{-x}}{2}=\cos ix
\end{equation}
\begin{equation}
    \tanh x = \frac{\sinh x}{\cosh x} = \frac{e^x-e^{-x}}{e^x+e^{-x}}
\end{equation}
\begin{equation}
    \cosh^2 x - \text{sinh}^2 x = 1
\end{equation}
\begin{equation}
    \tanh^2 x + \text{sech}^2 x = 1
\end{equation}

\subsection{Series and Expansions}

\noindent
\textbf{Fourier Series}
\small
\begin{equation}
\begin{split}
    f(x) &= \frac{a_0}{2}\sum_{m=1}^\infty\bigg[a_m\cos{\bigg(\frac{m\pi x}{L}\bigg)} + b_m\sin{\bigg(\frac{m\pi x}{L}\bigg)}\bigg] \\
    &= \sum_{n=-\infty}^{\infty}c_n \exp{\bigg(\frac{in\pi x}{L}\bigg)}
\end{split}
\end{equation}
\normalsize

\noindent
where: 
\begin{equation}
    a_0 = \frac{1}{L}\int_{-L}^Lf(x)\ dx
\end{equation}
\begin{equation}
    a_n = \frac{1}{L}\int_{-L}^Lf(x)\cos{\bigg(\frac{n\pi x}{L}\bigg)}\ dx
\end{equation}
\begin{equation}
    b_n = \frac{1}{L}\int_{-L}^Lf(x)\sin{\bigg(\frac{n\pi x}{L}\bigg)}\ dx
\end{equation}
\begin{equation}
    c_n = \frac{1}{2L}\int_{-L}^Lf(x)\exp{\bigg(-\frac{in\pi x}{L}\bigg)}\ dx
\end{equation} \newline
\noindent
\textbf{Taylor Series}
\begin{equation}
    f(x) = \sum_{n=0}^\infty \frac{f^n(a)}{n!}(x-a)^n
\end{equation}

\noindent
\textbf{Maclaurin Series} \newline
\noindent
A Taylor series centred at zero.

\begin{equation}
    f(x) = \sum_{n=0}^\infty \frac{f^n(0)}{n!}(x)^n
\end{equation}

\noindent
\textbf{Binomial Expansion}
\small
\begin{equation}
    (a+b)^n &= a^n + \begin{pmatrix}n\\1\end{pmatrix}a^{n-1}b + ... +  \begin{pmatrix}n\\r\end{pmatrix}a^{n-r}b^r + ... + b^n 
\end{equation}
\normalsize

\noindent
where:
\begin{equation}
    \begin{pmatrix}n\\r\end{pmatrix} = ^nC_r = \frac{n!}{r!(n-r)!}
\end{equation}

\subsection{Transforms}

\noindent
\textbf{Fourier Transform}
\begin{equation}
    \mathcal{F}[f(x)] = \Tilde{f}(k) = \frac{1}{\sqrt{2\pi}}\int_{-\infty}^\infty f(x)e^{-ikx}\ dx
\end{equation}

\noindent
\textbf{Inverse Fourier Transform}
\begin{equation}
    \mathcal{F}^{-1}[\Tilde{f}(k)] = f(x) = \frac{1}{\sqrt{2\pi}}\int_{-\infty}^\infty \Tilde{f}(k)e^{ikx}\ dk
\end{equation}

\noindent
\textbf{Fourier Transform Properties}
\begin{itemize}
    \item{$\mathcal{F}[af(x) + bg(x)] = a\Tilde{f}(k) + b\Tilde{g}(k)$}
    \item{$\mathcal{F}[f(ax)] = \frac{1}{|a|}\Tilde{f}\big(\frac{k}{a}\big)$}
    \item{$\mathcal{F}[f'(x)] = ik\Tilde{f}(k)$}
    \item{$\mathcal{F}[f(x-x_0)] = e^{-ikx_0}\Tilde{f}(k)$}
    \item{$\int_{-\infty}^\infty |f(x)|^2\ dx = \int_{-\infty}^\infty |\Tilde{f}(k)|^2\ dk$}
\end{itemize}

\noindent
\textbf{Laplace Transform}
\begin{equation}
    \Tilde{f}(s) = \int_0^\infty f(t)e^{-st}\ dt
\end{equation}

\noindent
\textbf{Laplace Transform Properties}
\begin{itemize}
    \item{$L[f^{'}(t)] = -f(0) + sL[f(t)]$} 
    \item{$L[f^{n}(t)] = -f(0) + s^nL[f(t)] - s^{n-1}f(0) - s^{n-2}\dot{f}(0) - ... - \frac{d^nf(0)}{dt^n}$} 
\end{itemize}

\subsection{Calculus}

\noindent
\textbf{L'Hôpital's Rule}
\begin{equation}
    \lim_{x \to a}\bigg(\frac{f(x)}{g(x)}\bigg) = \lim_{x \to a}\bigg(\frac{f'(x)}{g'(x)}\bigg)
\end{equation}
\noindent
This is useful when $f(a) = g(a) = 0$ but $f'(a) \neq 0$ and $g'(a) \neq 0$. \newline

\noindent
\textbf{Chain Rule}
\begin{equation}
    \frac{df(u(x))}{dx} = \frac{du}{dx}\frac{df}{du}
\end{equation}

\noindent
\textbf{Product Rule}
\begin{equation}
    \frac{df(u(x)v(x))}{dx} = u\frac{dv}{dx} + v\frac{du}{dx}
\end{equation}

\noindent
\textbf{Quotient Rule}
\begin{equation}
    \frac{df\big(\frac{u(x)}{v(x)}\big)}{dx} = \frac{v\frac{du}{dx} - u\frac{dv}{dx}}{v^2}
\end{equation}

\noindent
\textbf{Partial Differentiation}
\begin{equation}
    df(x,y) = \frac{\partial f}{\partial x}dx + \frac{\partial f}{\partial y}dy
\end{equation}

\noindent
\textbf{Reciprocal Theorem}
\begin{equation}
    \frac{\partial x}{\partial z} = \frac{1}{\frac{\partial z}{\partial x}}
\end{equation}

\noindent
\textbf{Reciprocity Theorem}
\begin{equation}
    \frac{\partial x}{\partial y}\frac{\partial y}{\partial z}\frac{\partial z}{\partial x} = -1
\end{equation}

\noindent
\textbf{Jacobian}
\footnotesize
\begin{equation}
    \iiint_D{f(x,y,z)}\ dxdydz \to \iiint_D{f(u,v,w)}\ |\mathbf{J}|dudvdw
\end{equation}
\normalsize
\begin{equation}
    \mathbf{J} =
    \begin{vmatrix}
    \frac{\partial x}{\partial u} & \frac{\partial x}{\partial v} & \frac{\partial x}{\partial w} \\
    \frac{\partial y}{\partial u} & \frac{\partial y}{\partial v} & \frac{\partial y}{\partial w} \\
    \frac{\partial z}{\partial u} & \frac{\partial z}{\partial v} & \frac{\partial z}{\partial w} \\
    \end{vmatrix}
\end{equation}

\noindent
\textbf{Curve Length}
\begin{equation}
\begin{split}
    C &= \int_a^b\sqrt{1 + \bigg(\frac{dy}{dx}\bigg)^2}\ dx \\
    &= \int_a^b\sqrt{\bigg(\frac{dx}{dt}\bigg)^2 + \bigg(\frac{dy}{dt}\bigg)^2}\ dt 
\end{split}
\end{equation}

\noindent
\textbf{Area Under Curves}
\begin{equation}
\begin{split}
    A &= \int_{x_1}^{x_2} F[x, y(x)]\sqrt{1 + \bigg(\frac{dy}{dx}\bigg)^2}\ dx \\
    &= \int_{y_1}^{y+2} F[x(y), y]\sqrt{\bigg(\frac{dx}{dy}\bigg)^2 + 1}\ dy \\
    &= \int_{t=a}^{t=b} f[x(t), y(t)]\sqrt{\bigg(\frac{dx}{dt}\bigg)^2 + \bigg(\frac{dy}{dt}\bigg)^2}\ dt
\end{split}
\end{equation}

\noindent
\textbf{Green's Theorem}
\begin{equation}
    \oint_c \big(Pdx + Qdy\big) = \iint \bigg(\frac{\partial Q}{\partial x} - \frac{\partial P}{\partial y}\bigg)\ dA
\end{equation}

\noindent
\textbf{Area of Surface}
\begin{equation}
    A = \iint_S \sqrt{1 + \bigg(\frac{dz}{dx}\bigg)^2 + \bigg(\frac{dz}{dy}\bigg)^2}\ dxdy    
\end{equation}

\noindent
\textbf{Dirac Delta Function}
\begin{equation}
    f(X) = \int_\infty^\infty f(x)\delta(x-X)dx
\end{equation}

\noindent
\textbf{Properties of the Dirac Delta Function}
\begin{itemize}
    \item{Symmetry: $\delta(-x) = \delta(x)$}
    \item{$\delta[a(x-X)] = \frac{1}{|a|}\delta(x-X)$}
    \item{$\int_\infty^\infty\delta'(x)f(x)dx = -f'(0)$}
    \item{$\frac{d}{dx}\delta(x) = -\frac{1}{x}\delta(x)$}
\end{itemize}

\noindent
\textbf{Trigonometric Functions}

\renewcommand{\arraystretch}{2}
\begin{center}
\begin{supertabular}{|P{3cm}|P{3cm}|}
    \hline
    \textbf{Function} & \textbf{Derivative} \\ \hline
    $\sin(x)$ & $\cos(x)$ \\ \hline
    $\cos(x)$ & $-\sin(x)$ \\ \hline
    $\tan(x)$ & $\sec^2(x)$ \\ \hline
    $\csc(x)$ & $-\cot(x)\csc(x)$ \\ \hline
    $\sec(x)$ & $\sec(x)\tan(x)$ \\ \hline
    $\cot(x)$ & $-\csc^2(x)$ \\ \hline
    $\arcsin(x)$ & $\frac{1}{\sqrt{1-x^2}}$ \\ \hline
    $\arccos(x)$ & $-\frac{1}{\sqrt{1-x^2}}$ \\ \hline
    $\arctan(x)$ & $\frac{1}{\sqrt{1+x^2}}$ \\ \hline
    $\arccsc(x)$ & $-\frac{1}{|x|\sqrt{x^2-1}}$ \\ \hline
    $\arcsec(x)$ & $\frac{1}{|x|\sqrt{x^2-1}}$ \\ \hline
    $\arccot(x)$ & $-\frac{1}{\sqrt{1+x^2}}$ \\ \hline
\end{supertabular} \newline
\end{center}

\noindent
\textbf{Hyperbolic Functions}

\renewcommand{\arraystretch}{2}
\begin{center}
\begin{supertabular}{|P{3cm}|P{3cm}|}
    \hline
    \textbf{Function} & \textbf{Derivative} \\ \hline
    $\sinh(x)$ & $\cosh(x)$ \\ \hline
    $\cosh(x)$ & $\sinh(x)$ \\ \hline
    $\tanh(x)$ & $\sech^2(x)$ \\ \hline
    $\csch(x)$ & $-\coth(x)\csch(x)$ \\ \hline
    $\sech(x)$ & $-\sech(x)\tanh(x)$ \\ \hline
    $\coth(x)$ & $-\csch^2(x)$ \\ \hline
    $\arcsinh(x)$ & $\frac{1}{\sqrt{x^2+1}}$ \\ \hline
    $\arccosh(x)$ & $\frac{1}{\sqrt{x^2-1}}$ \\ \hline
    $\arctanh(x)$ & $\frac{1}{\sqrt{1-x^2}}$ \\ \hline
    $\arccsch(x)$ & $-\frac{1}{|x|\sqrt{1+x^2}}$ \\ \hline
    $\arcsech(x)$ & $-\frac{1}{x\sqrt{1-x^2}}$ \\ \hline
    $\arccoth(x)$ & $\frac{1}{\sqrt{1-x^2}}$ \\ \hline
\end{supertabular}
\end{center}
\renewcommand{\arraystretch}{2}

\subsection{Vector Calculus}

\noindent
\textbf{Directional Gradient}
\begin{equation}
    \text{Directional Gradient} = \frac{\nabla\phi \cdot \mathbf{a}}{|\mathbf{a}|}
\end{equation}

\noindent
\textbf{Divergence Theorem}
\begin{equation}
    \iiint_V{(\nabla \cdot \textbf{A})dV} = \oiint_S{(\textbf{A} \cdot \textbf{n}) dS}
\end{equation}

\noindent
\textbf{Stoke's Theorem}
\begin{equation}
    \oint_l{\textbf{A} \cdot d\textbf{l}} = \iint_S{(\nabla \times \textbf{A}) \cdot d\textbf{S}}
\end{equation}

\noindent
\textbf{Convolution Theorem}
\begin{equation}
    \mathcal{F}(f \otimes g) = \sqrt{2\pi}\mathcal{F}(f)\mathcal{F}(g)
\end{equation}

\subsection{Ordinary Differential Equations}

\noindent
\textbf{Separate Variables}
\begin{equation}
    \frac{dy}{dx} = g(x)h(y)
\end{equation}
\begin{equation}
    \int\frac{dy}{h(y)} = \int g(x)\ dx
\end{equation}

\noindent
\textbf{Homogeneous}
\begin{equation}
    \frac{dy}{dx} = f\bigg(\frac{y}{x}\bigg) = f(v)
\end{equation}
\begin{equation}
    \frac{dy}{dx} = x\frac{dv}{dx} + v = f(v)
\end{equation}

\noindent
\textbf{Exact Equation}
\begin{equation}
    P(x, y)dx + Q(x, y)dy = 0,
\end{equation}
\noindent
where:
\begin{equation}
    P(x, y) = \frac{\partial F}{\partial x}\bigg|_y\ \ \text{and}\ \ Q(x, y) = \frac{\partial F}{\partial y}\bigg|_x.
\end{equation}
\noindent
Equation is exact if:
\begin{equation}
    \frac{\partial P}{\partial y} = \frac{\partial Q}{\partial x}.
\end{equation}
\noindent 
Solve for $F$ by integrating $P$ and $Q$ and define constants so that the equations match. Rearrange to get $y$ in terms of $x$. \newline

\noindent
\textbf{Particular Integrals}
\begin{equation}
    \frac{dy}{dx} + P(x)y = Q(x)
\end{equation}
\begin{equation}
    I = e^{\int{P(x)dx}} 
\end{equation}
\begin{equation}
    \frac{d}{dx}(Iy) = IQ
\end{equation}
\noindent
Solve for y. \newline

\begin{sidewaystable*}
\subsection{Coordinate Systems}
\centering
\def\arraystretch{1.5}
\resizebox{\textwidth}{!}{\begin{tabular}{|c|c|c|c|}

\hline%------------------------------------------------------------------------
\textbf{Operation} & \textbf{Cartesian $(x,y,z)$}	& \textbf{Cylindrical $(\rho,\phi,z)$} &	\textbf{Spherical $(r,\theta,\phi)$}
\\
\hline%------------------------------------------------------------------------
\multirow{\textbf{Definition}} & $\displaystyle x=x$
 & $\displaystyle x=\rho\cos\phi $ & $\displaystyle x=r\sin\theta\cos\phi $\\
 &$\displaystyle y=y$ & $\displaystyle y=\rho\sin\phi$  & $x=r\sin\theta\cos\phi $\\
& $\displaystyle z=z$ & $\displaystyle z=z$ & $\displaystyle z=r\cos\theta$\\ 
\hline %------------------------------------------------------------------------
&&&\\[-0.5cm]

\multirow{\textbf{Unit Vectors}} & $\displaystyle \hat{\boldsymbol{\rho}}=\frac{x\hat{\mathbf{x}}+y\hat{\mathbf{y}}}{\sqrt{x^2+y^2}}$
 &$\hat{\mathbf x} =\cos\phi\hat{\boldsymbol{\rho}} - \sin\phi\boldsymbol{\hat{\phi}}$ & $\hat{\mathbf x} = \sin\theta\cos\phi\boldsymbol{\hat{r}} + \cos\theta\cos\phi\boldsymbol{\hat{\theta}}-\sin\phi\boldsymbol{\hat{\phi}}  $\\
 
 &$\displaystyle \hat{\boldsymbol{\phi}}=\frac{-y\hat{\mathbf{x}}+x\hat{\mathbf{y}}}{\sqrt{x^2+y^2}}$
  & $\hat{\mathbf y} = \sin\phi\boldsymbol{\hat{\rho}} + \cos\phi\boldsymbol{\hat{\phi}}$  & $\hat{\mathbf y} = \sin\theta\sin\phi\boldsymbol{\hat{r}} + \cos\theta\sin\phi\boldsymbol{\hat{\theta}}+\cos\phi\boldsymbol{\hat{\phi}} $\\
 
 &$\mathbf{\hat{r}}         = \displaystyle\frac{x \hat{\mathbf x} + y \hat{\mathbf y} + z \mathbf{\hat{z}}}{\sqrt{x^2+y^2+z^2}}$ & $\unit{z}=\unit{z}$ & $\displaystyle \boldsymbol{\hat{\theta}} = \frac{x z \hat{\mathbf x} + y z \hat{\mathbf y} - \left(x^2 + y^2\right) \mathbf{\hat{z}}}{\sqrt{x^2+y^2} \sqrt{x^2+y^2+z^2}} $
\\&&&
\\[-0.5cm]
\hline %------------------------------------------------------------------------
&&&\\[-0.5cm]
\textbf{Grad ($\nabla f$)}
 & $\displaystyle{\partial f \over \partial x}\hat{\mathbf x} + {\partial f \over \partial y}\hat{\mathbf y}
 + {\partial f \over \partial z}\mathbf{\hat{z}}$
  & $\displaystyle{\partial f \over \partial \rho}\boldsymbol{\hat{\rho}}
  + {1 \over \rho}{\partial f \over \partial \phi}\boldsymbol{\hat{\phi}}
  + {\partial f \over \partial z}\mathbf{\hat{z}}$
   & $\displaystyle{\partial f \over \partial r}\boldsymbol{\hat{r}}
   + {1 \over r}{\partial f \over \partial \theta}\boldsymbol{\hat{\theta}}
   + {1 \over r\sin\theta}{\partial f \over \partial \phi}\boldsymbol{\hat{\phi}}$
   \\[0.3cm]
\hline%------------------------------------------------------------------------
&&&\\[-0.5cm]
\textbf{Div ($\nabla \cdot \boldsymbol{a}$)} & $\displaystyle{\partial A_x \over \partial x} + {\partial A_y \over \partial y} + {\partial A_z \over \partial z}$ &$\displaystyle{1 \over \rho}{\partial \left( \rho A_\rho  \right) \over \partial \rho}
+ {1 \over \rho}{\partial A_\phi \over \partial \phi}
+ {\partial A_z \over \partial z}$& $\displaystyle{1 \over r^2}{\partial \left( r^2 A_r \right) \over \partial r}
+ {1 \over r\sin\theta}{\partial \over \partial \theta} \left(  A_\theta\sin\theta \right)
+ {1 \over r\sin\theta}{\partial A_\phi \over \partial \phi}$\\[0.3cm]
\hline%------------------------------------------------------------------------
& & &\\[-0.5cm]
\textbf{Curl ($\nabla \times \boldsymbol{a}$)} &
$\displaystyle \left | \begin{array}{c c c}
\boldsymbol{\hat{x}} & \boldsymbol{\hat{y}} & \boldsymbol{\hat{z}}\\
\displaystyle\pardif{}{x} & \displaystyle\pardif{}{y} & \displaystyle\pardif{}{z}\\
a_x & a_y & a_z\\
\end{array}\right|$ &
$\displaystyle \frac{1}{\rho}\left | \begin{array}{c c c}
\boldsymbol{\hat{\rho}} & \rho\boldsymbol{\hat{\phi}} & \boldsymbol{\hat{z}}\\
\displaystyle\pardif{}{\rho} & \displaystyle\pardif{}{\phi} & \displaystyle\pardif{}{z}\\
a_\rho & \rho a_\phi & a_z\\

\end{array}\right|$ & 
$\displaystyle \left | \begin{array}{c c c}
\displaystyle\frac{\boldsymbol{\hat{r}}}{r^2\sin\theta}
& \displaystyle\frac{\boldsymbol{\hat{\theta}}}{r\sin\theta} & \displaystyle\frac{\boldsymbol{\hat{\phi}}}{r}\\
\displaystyle\pardif{}{r} & \displaystyle\pardif{}{\theta} & \displaystyle\pardif{}{\phi}\\
a_r & r a_\theta & r\sin\theta a_\phi\\

\end{array}\right|$\\&&&
\\[-0.5cm]
\hline%------------------------------------------------------------------------
\textbf{Area Element (d$\boldsymbol{A}$)} & d$x$d$y\boldsymbol{\hat{z}}$ & $\rho$d$\rho$d$\phi\boldsymbol{\hat{z}}$ & $r^2\sin\theta$d$\theta$d$\phi\boldsymbol{\hat{r}}$\\
\hline%------------------------------------------------------------------------
\textbf{Volume Element (d$V$)} & d$x$d$y$d$z$ & $\rho$d$\rho$d$\phi$d$z$ & $r^2\sin\theta$d$\theta$d$\phi$d$r$\\
\hline%------------------------------------------------------------------------
\end{tabular}}
\end{sidewaystable*}

\end{document}
